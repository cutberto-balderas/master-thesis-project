\chapter{Introducción}
\label{chap:introduction}
\textit{En la presente sección se hablara a grandes rasgos sobre el tema de interés de la tesis
y las técnicas para estudiarlo.}
\vfill
\minitoc
\newpage

\section{Motivación del trabajo}
\label{ch1-why}
Este trabajo tiene como finalidad entender las propiedades de los semiconductores $CdTe(001)$ y 
$ Hg_{0.18}Cd_{0.82}Te (001)$, estos siendo sometidos a un estrés o tensión, en el caso del primero 
la deposición de una película delgada de Ag y del segundo, el tallar mecánicamente su superficie con una 
pasta de diamante. Esto con la finalidad de establecer un \textit{estándar} para comprender las propiedades 
de los materiales llamados \textit{aislantes topológicos}, aunque ninguno de los materiales estudiados están 
en este grupo, sus compuestos y heteroestructuras si presentan las propiedades exóticas. Además se realizo 
la implementación de programas de control para el monocromador HR60.

\section{Aislantes Topológicos}
\label{ch1-ti}
Los aislantes topológicos se caracterizan por estados exóticos metálicos, en las cuales los electrones viajan insensibles 
a las colisiones producidas por las impurezas\cite{Moore2010}\cite{Konig2007}. La idea sobre los aislantes topológicos 
surgió del trabajo del efecto Hall Cuántico. Este efecto que se presenta en estructuras semiconductoras de dos 
dimensiones bajo un campo magnético externo\cite{Klitzing1980}. Una traza de este efecto debe de observarse aun sin la 
presencia de un campo magnético externo considerando que en un semiconductor los electrones al moverse dentro del 
mismo, sentirán un campo magnético efectivo producido por los núcleos positivos del cristal (el llamado acople 
espín-orbita o \textit{SOC},por sus siglas en ingles). Las interfaces metal/semiconductor de materiales con un fuerte 
\textit{SOC} presentan el \textit{efecto Rashba} que aunque no son aislantes topológicos como tal, presentan 
algunos fenómenos de los mismos, como la formación de corrientes Hall en su superficie.\cite{Lin2014}\cite{Bihlmayer2015}

La forma mas simple de describir un aislante topológico es como un aislante cuya frontera con el vacío siempre permanece 
metálica. Para que un aislante topológico se forme, la interacción espín-orbita debe ser fuerte y modificar 
significativamente la estructura electrónica. Esto sugiere que los semiconductores de brecha fundamental muy 
pequeña sean muy adecuados para configurar aislantes topológicos. Dentro de los materiales con fases topológicas 
podemos mencionar la aleación $Bi_{x}Sb_{1-x}$ y el semimetal $HgTe$\cite{Konig2007}\cite{Moore2010}.

\section{Técnicas Ópticas de Caracterización}
\label[type]{ch1-opt-charac-tech}
Las técnicas ópticas de caracterización, utilizan la luz como sonda para obtener diferentes propiedades de 
los materiales, que van desde su morfología y estructura molecular, hasta de carácter electrónico, de 
forma segura, ya que son técnicas no destructivas que pueden aplicarse tanto \textit{in-situ} o 
\textit{ex-situ}. En el presente trabajo, se busca entender las propiedades de los materiales de interés 
antes mencionados, por medio de las siguientes técnicas.

\begin{itemize}
    \item Reflectancia diferencial espectroscópica \textit{(RDS)}. Esta técnica es sensible al rompimiento 
    de simetría. En semiconductores o estructuras cubicas la RDS da información del estado de la superficie 
    o de las interfaces. Esta técnica es una herramienta potencial para el estudio de las interfaces 
    aislante/metal que se produce en los aislantes topológicos.

    \item Espectroscopia Raman. En este trabajo, se utilizara para caracterizar las interfaces 
    metal/semiconductor, ya que esta técnica sondea la estructura del material utilizando un haz de luz 
    monocromático. Además de darnos una idea del estrés relativo al que esta sometido el material, siendo 
    un parámetro importante para el estudio de los aislantes topológicos.

    \item Microscopia de Fuerza Atómica \textit{(AFM)} y Microscopia Óptica de Campo Cercano 
    \textit{(NSOM)}. La unión de ambas técnicas nos da una idea de la morfología por medio de \textit{AFM}, 
    utilizando las fuerzas atractivas y repulsivas en la interaccion sistema/muestra y la respuesta óptica 
    del material por medio de \textit{NSOM}, por lo que podemos caracterizar superficies y complementar 
    las otras técnicas.
\end{itemize}
