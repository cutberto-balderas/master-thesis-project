\chapter{Conclusiones y trabajo a futuro}
\label{chap:conclusions}
\textit{}
\vfill
\minitoc
\newpage
\section{Conclusiones y trabajo a futuro}
\label{ch5:conlusions}
Se implementaron y realizaron experimentos de reflectancia diferencial (RD) sobre cristales de CdTe(001) y 
Cd0.18Hg0.82Te(001). En el caso de CdTe se realizaron evaporaciones de Ag con el objetivo de caracterizar la respuesta 
óptica de una interfaz de Ag/CdTe. Esto tiene el objetivo de poder identificar las componentes y evolución de la RD de 
una interfaz metal/semiconductor y probar que con la RD es posible identificar la respuesta metálica de la superficie.
Adicionalmente, se realizaron mediciones de Raman para la caracterización de la capa de Ag depositada y se contrastaron
contra un sustrato sin deposición, donde se observo un estrés causado por la diferencia entre los parámetros de red 
de los materiales, tambien fue posible determinar que la interfaz metal/semiconductor es de buena calidad y no presenta 
daños o gran desorden. Usando AFM y NSOM se observo que aunque las muestras no eran rugosas, teníamos una diferencia en
la respuesta óptica del material, debido a la diferencia en la morfología del mismo.

De igual forma se midió RD a superficies de Cd0.18Hg0.82Te(001) con tensión y sin tensión superficial. Esta tensión fue 
generada por medio del tallado mecánico de la superficie con pasta de diamante. Considerando que el Cd0.18Hg0.82Te(001) 
tiene propiedades de semimetal en bulto, la tensión superficial que se extiende algunas micras al bulto, rompe la 
degeneración abriendo la brecha fundamental produciendo una interface aislante/metal que puede ser caracterizada por RD. 
Se demuestra la gran sensibilidad que la RD tiene en la caracterización de las interfaces bajo estudio.

En el presente trabajo, se contribuyó a la identificación y comportamiento de algunas componentes de la RD 
fundamentales en el entendimiento de las interfaces metal/semiconductor. Como trabajo a futuro se planea realizar 
mas experimentos de RD sobre el Cd0.18Hg0.82Te y heteroestructuras de este material, con el objetivo de poder 
establecer una relación entre los espectros de RD y el comportamiento topológico de estos materiales.

Se implementaron programas de control para el monocromador HR60, para recibir instrucciones desde el equipo de control, 
para el control del motor principal que controla el cambio en la longitud de onda, el movimiento de las cuatro slits y 
el cambio de la resolución desde 1 nm hasta 0.1 nm.
