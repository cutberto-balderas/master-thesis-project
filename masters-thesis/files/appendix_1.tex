\appendix

\chapter{Programas de control}
\label{chap:control}
\textit{En este apartado se colocaran programa desarrollados para el control de
los equipos del laboratorio de Elipsometria Espectroscopia.}
\vfill
\minitoc
\newpage

\section{Codigo en Arduino del monocromador HR60}
\label{appendix:arduino}

En este apartado dejaremos constancia del programa de Arduino para el control del monocromador HR60 discutido en la 
seccion \ref{ch2:monochromator}.\\

\begin{lstlisting}[style=Arduino]
    // Para mas informacion sobre las librerias utilizadas consultar 
    // https://learn.adafruit.com/adafruit-motor-shield-v2-for-arduino/overview, 
    // libreria oficial del Stepper Motor Shield proporcionado por Adafruit.
    
    #include <Wire.h>
    #include <Adafruit_MotorShield.h>
    #include "utility/Adafruit_MS_PWMServoDriver.h"
    
    Adafruit_MotorShield AFMS1(0x60);  // Creamos el shield de los motores de las slits, bottom shield.
    Adafruit_MotorShield AFMS2(0x61);  // Creamos el shield de los motores de las slits, top shield.
    
    Adafruit_StepperMotor *myMotor_SIRF = AFMS1.getStepper(200, 2);   // Creamos el objeto para mover la slit SIRF
    Adafruit_StepperMotor *myMotor_SELF = AFMS2.getStepper(200, 2);   // Creamos el objeto para mover la slit SELF
    
    Adafruit_StepperMotor *myMotor_SILF = AFMS1.getStepper(200, 1);   // Creamos el objeto para mover la slit SILF
    Adafruit_StepperMotor *myMotor_SELL = AFMS2.getStepper(200, 1);   // Creamos el objeto para mover la slit SELL
    
    void setup() {
      AFMS1.begin();    //  Inicializamos ambas shields que controlan las slits del monocromador.
      AFMS2.begin();  
      
      //  Definiremos las entradas y salidas logicas para el control del motor.
    
        pinMode(2, OUTPUT); //  pin de salida para el control de la direccion del motor, DIR
        pinMode(3, OUTPUT); //  pin de salida para el control del movimiento del motor, STEP 
        pinMode(4, OUTPUT); //  pin de salida para el control del paso del monocromador, MS1
        pinMode(5, OUTPUT); //  pin de salida para el control del paso del monocromador, MS2
        pinMode(6, OUTPUT); //  pin de salida para el control del paso del monocromador, MS3
    
        pinMode(7, INPUT);  //  pin de entrada para el limit switch en el motor del espejo.
    
        // Definiremos la configuracion del motor en la fraccion de paso utilizado, para esto,
        // debemos colocar los pines como se muestra a continuacion,
        // siendo H - High y L - Low en el estado digital.
    
        // MS1 MS2 MS3 Microstep Resolution
        // L   L   L   Full Step
        // H   L   L   Half Step          Estamos trabajando en esta resolucion!
        // L   H   L   Quarter step
        // H   H   L   Eighth step
        // H   H   H   Sixteenth step
    
        digitalWrite(4, HIGH);
        digitalWrite(5, LOW);
        digitalWrite(6, LOW);
    
      myMotor_SIRF->setSpeed(20);     //  Definimos la velocidad de los motores que controlan las slits, esta esta en revoluciones
      myMotor_SELF->setSpeed(20);     //  por minuto, lo que equivaldria a 4000 pasos por minuto.
      myMotor_SILF->setSpeed(20);
      myMotor_SELL->setSpeed(20);
      
      //  Iniciaremos la comunicacion serial a 9600 bits/s
        Serial.begin(9600);
    }
    
      
    void loop() {// Definimos el tamano del paso que tomara el motor
     
     //  Primero debemos checar comunicacion entre el Arduino y la PC
     String lectura;
     float lecturai;
     int i;
     int j;
     int f_r = 10;
     
     if (Serial.available() != 0){
      switch(Serial.read()){
        case 'F':                                     //Moviemiento del monocromador, este tiene una resolucion de Full step ~0.1 nm
              lectura = Serial.readString();
              lectura.replace(String(char(13)), "");
              lectura.trim();
              
              lecturai = 0;
              lecturai = lectura.toFloat();
    
              Serial.print('M');    //  Indicamos que el monocromador esta en movimiento.
              
              // Definimos la direccion en la que el monocromador se movera.
                if(lecturai > 0){
                  lecturai = abs(lecturai);
                    digitalWrite(2, HIGH);
                }
                else{
                  lecturai = abs(lecturai);
                    digitalWrite(2, LOW);
                }
              //  Hacemos los pasos en el motor del monocromador, hasta que cumplamos 
              //  el numero de pasos requerido o topemos con el limit switch.
                i = 1;
                while(f_r*lecturai >= i){
                  j=0;
                  while(j < 2){
                    digitalWrite(3, HIGH);
                    delayMicroseconds(450);
                    digitalWrite(3, LOW);
                    delayMicroseconds(450);
                    delay(5);
                    Serial.print(j);
                    Serial.write(';');
                    j++;
                  }
                  i++;
                  Serial.print(i);
                  Serial.print('\n');
                }      
            Serial.write('o');
       break;
    
       case 'A':                          //  Inicializacion del monocromador, lo mandamos hasta el extremo y vemos cuando topa con
                                          //  el limit switch.
            Serial.print('M');    //  Indicamos que el motor esta en movimiento
              i = 1;
              while(digitalRead(7) == LOW){
                digitalWrite(2, HIGH);
         
                digitalWrite(3, HIGH);
                delayMicroseconds(500);
                digitalWrite(3, LOW);
                delayMicroseconds(500);
                delay(5);
    
                i++;              //  Verificar el numero de pasos dados durante la iteracion.
                Serial.print(i);
                Serial.print('\n');
              }
Serial.write('o');//Mandamos la senal que el motor termino su movimiento y esta listo para el siguiente.
      break;
    
      // Codigo para el movimiento de las slits. Dependiendo cuales se busca mover, utilizaremos:
      // SILR: 'G', SIFR: 'H', SEFL: 'K', SELL: 'J', 
    
      case 'H':     // SIFR: 'H'
              lectura = Serial.readString();
              lectura.replace(String(char(13)), "");
              lectura.trim();
              
              lecturai = 0;
              lecturai = (lectura.toInt())/5;
              
              Serial.print('S');    //  Indicamos que las slits esta en movimiento
              
              // Definiremos el movimiento de las slits y su direccion.
              if (lecturai > 0){
                myMotor_SIRF->step(lecturai, FORWARD, SINGLE);
              }
              else{
                myMotor_SIRF->step(abs(lecturai), BACKWARD, SINGLE);
                }
          Serial.write('o');
      break;
    
      case 'K':     // SEFL: 'K'
              lectura = Serial.readString();
              lectura.replace(String(char(13)), "");
              lectura.trim();
              
              lecturai = 0;
              lecturai = (lectura.toInt())/5;
              
              Serial.print('S');    //  Indicamos que las slits esta en movimiento
              
              // Definiremos el movimiento de las slits y su direccion.
              if (lecturai > 0){
                myMotor_SELF->step(lecturai, FORWARD, SINGLE);
              }
              else{
                myMotor_SELF->step(abs(lecturai), BACKWARD, SINGLE);
                }
          Serial.write('o');
      break;
    
      case 'G':     // SILR: 'G'
              lectura = Serial.readString();
              lectura.replace(String(char(13)), "");
              lectura.trim();
              
              lecturai = 0;
              lecturai = (lectura.toInt())/5;
              
              Serial.print('S');    //  Indicamos que las slits esta en movimiento
              
              // Definiremos el movimiento de las slits y su direccion.
              if (lecturai > 0){
                myMotor_SILF->step(lecturai, FORWARD, SINGLE);
              }
              else{
                myMotor_SILF->step(abs(lecturai), BACKWARD, SINGLE);
                }
        Serial.write('o');
      break;
    
      case 'J':     // SELL: 'J'
              lectura = Serial.readString();
              lectura.replace(String(char(13)), "");
              lectura.trim();
              
              lecturai = 0;
              lecturai = (lectura.toInt())/5;
              
              Serial.print('S');    //  Indicamos que las slits esta en movimiento
            // Definiremos el movimiento de las slits y su direccion.
              if (lecturai > 0){
                myMotor_SELL->step(lecturai, FORWARD, SINGLE);
              }
              else{
                myMotor_SELL->step(abs(lecturai), BACKWARD, SINGLE);
                }
        Serial.write('o');
      break;
      
      case 'R':      // La resolucion del paso del monocromador es 0.1 nm, usando half step estamos dando dos pasos de 0.05 nm
              lectura = Serial.readString();
              lectura.replace(String(char(13)), "");
              lectura.trim();
    
              if(lectura.toInt() == 1){ // Es el caso para la resolucion de 1 nm, repetiremos el paso 10 veces.
                f_r = 10;
              }
              if(lectura.toInt() == 2){ // Es el caso para la resolucion de 0.5 nm, repetiremos el paso 5 veces.
                f_r = 5;
              }
              if(lectura.toInt() == 3){ // Es el caso para la resolucion de 0.2 nm, repetiremos el paso 2 veces.
                f_r = 2;
              }
              if(lectura.toInt() == 4){ // Es el caso para la resolucion de 0.1 nm, repetiremos el paso 1 veces.
                f_r = 1;
              }
        Serial.write('o');
      break;
      }
     }
    }
\end{lstlisting}