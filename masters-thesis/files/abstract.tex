\chapter{Resumen}
\label{chap:abstract}

Los aislantes topológicos son materiales que tienen un comportamiento de aislantes en el bulto y 
de metal en su superficie. Actualmente, existe un gran interés en el estudio de sus propiedades 
físicas tanto superficiales y de bulto. Los sistemas formados por heteroestructura basadas en 
materiales como el $ Hg_{1-x}Cd_{x}Te $ han mostrado tener propiedades topológicas y por lo tanto 
han despertado un gran interés para su aplicación en este campo. Dentro de las técnicas más 
adecuadas para caracterizar aislantes topológicos se encuentran las técnicas ópticas. En 
particular con la Espectroscopia de Reflectancia Diferencial que es sensible al estado de la 
superficie o interface del sistema bajo estudio. En el presente trabajo de tesis, se realizaron 
estudios por medio de las técnicas Espectroscopia de Reflectancia Diferencial, Espectroscopia 
Raman, Microscopia de Fuerza Atómica y Microscopia Óptica de Campo Cercano a los cristales 
$ CdTe (001) $ y $ Hg_{0.18}Cd_{0.82}Te (001)$ bajo diferentes condiciones de su superficie. 
Además, se desarrollaron la instrumentación electrónica y de programación 
para la implementación y realización de los experimentos.