\usepackage[font=normalsize,labelfont=bf]{caption}
\captionsetup[table]{name=Tabla}
\usepackage[font=normalsize,labelformat=simple]{subcaption}
\renewcommand\thesubfigure{(\alph{subfigure})}
\usepackage{graphicx}
\usepackage[spanish, es-nodecimaldot]{babel}
\usepackage{circuitikz}
\usepackage{tabularx}
\usepackage[left=2.5cm,right=2.5cm,top=2.5cm,bottom=1cm,includefoot,includehead,headheight=15pt]{geometry}
\usepackage{miama}
\usepackage{subcaption }
\usepackage{calligra}
\usepackage[T1]{fontenc}
\usepackage[utf8]{inputenc}
\usepackage{amsmath,amssymb,amsfonts,latexsym,stmaryrd,eucal,amsthm,textcomp,cancel,verbatim,nccmath, mathtools} 
\allowdisplaybreaks
\usepackage{lmodern}
\usepackage{relsize}
\renewcommand{\_}{\textscale{.7}{\textunderscore}}
\usepackage{makeidx}
\makeindex
\usepackage{leftidx}
\usepackage{xcolor}
\usepackage{tikz,pgfplots}%,pgfarrows,pgfnodes,pgfautomata,pgfheaps,pgfshade}
\usetikzlibrary{calc,decorations,decorations.pathmorphing,arrows,matrix,positioning,patterns}
\usepackage{stanli}
\usepackage{pgfplotstable}
\usepackage[backref=page]{hyperref}
\usepackage{booktabs}
\usepackage{cleveref}
%\crefname{figure}{Figure}{Figures}
\usepackage{physics}        
\usepackage{mathrsfs}
\usepackage{xfrac}
\usepackage{units} % nicefrac
\usepackage[squaren]{SIunits} % provide \micro command 
\usepackage{sistyle}
%\usepackage[usenames, dvipsnames]{color}
\usepackage{array}
\usepackage{titlesec}
\usepackage{cite}
\usepackage{microtype}
\usepackage{enumerate}
\usepackage{blindtext}
\usepackage{floatrow}
\usepackage{placeins}
\usepackage{lettrine}
\usepackage{longtable}
\usepackage{calc}
\usepackage{pdfpages}
\usepackage[nomessages]{fp}% http://ctan.org/pkg/fp
\usepackage{scalerel}
\usepackage{ragged2e}
\usepackage{rotating}
\usepackage{hyperref}
\hypersetup{pdfborder=0 0 0}
\usepackage{datetime}
\newdateformat{monthyeardate}{\monthname[\THEMONTH] \THEYEAR}


%=================================== TOC MINITOC =========================%
\usepackage[spanish]{minitoc}
\usepackage{setspace}
\setstretch{1.1}
\usepackage[nottoc,notlof,notlot]{tocbibind}
\mtcindent=15pt
\setcounter{tocdepth}{3}
\setcounter{secnumdepth}{3}
\setcounter{minitocdepth}{3}
%\setlength{\heavyrulewidth}{5pt}
%\usepackage{cm-super}
\RequirePackage{fix-cm}
\usepackage{type1cm}
\usepackage{aecompl}
\usepackage{xargs}
\usepackage{caption}

\addto{\captionsspanish}{% Making babel aware of special titles
  \renewcommand{\mtctitle}{Contenido}
}

% Chapter display options
\titlespacing*{\chapter}{0cm}{-1.cm}{-40pt}%pbk
\titleformat{\chapter}[display]{\Huge\filleft\scshape}{\tikz{\draw node[scale=5,font=\bfseries,black] {\thechapter};}
}{20pt}{}[\titlerule\vspace{2ex}\filright\vspace{2ex}]
%
\titlespacing*{\part}{-10pt}{120pt}{-80pt}%pbk
\titleformat{\part}[frame]{\Huge\filcenter\scshape}{ \normalfont\bf\fontsize{85pt}{0pt}\selectfont \raisebox{1.7cm}{ Part\fontfamily{put}\fontseries{b}\fontsize{105pt}{0pt}\selectfont \hspace{.2em}\thepart} }{30pt}{}[\filright]

\newcommand\boxedSection[3]{{%
%
\begin{tikzpicture}[inner sep=#3,line width=1.0pt]
\node[anchor=east,rectangle,rounded corners=7,draw,fill=black!70!white] at (0,0) (counter) {\color{white}\textbf{#2}};
\draw (counter.south)  ++(.0pt,.5pt)-- ++($(\linewidth,0) - (25pt,0)$);
\draw ([yshift=-0.4mm]counter.north)  ++(.0pt,.5pt)-- ++($(\linewidth,0) - (25pt,0)$);
\node [right of=counter,anchor=west,yshift=-0.1mm]{#1};
\end{tikzpicture}
 }}
 \newcommand\boxedSectionB[3]
 {{%
%
\begin{tikzpicture}[inner sep=#3,line width=1pt]
%\node[anchor=east,rectangle,rounded corners=0.9,draw,fill=black] at (0,0) (counter) {\color{white}\textbf{#2}};
\node[anchor=east,rectangle,draw,fill=black] at (0,0) (counter) {\color{white}\textbf{#2}};
\draw (counter.south)++(.0pt,.5pt) --++($(0.955\textwidth,0) - (0pt,0)$);
%\draw (counter.north) ++(.0pt,.5pt)-- ++($(1.01\linewidth,0) - (2.9pt,0)$);
\node [right of=counter,anchor=west]{#1};
\end{tikzpicture}
 }}
\newcommand\boxedsection[1]{\boxedSectionB{#1}{\thesection}{2mm}}
\newcommand\boxedsubsection[1]{\boxedSection{#1}{\thesubsection}{1.7mm}}
\newcommand\boxedsubsubsection[1]{\boxedSection{#1}{\thesubsubsection}{1.5mm}}
 \titleformat{\section}[hang]%
     {\usekomafont{section}}%
     {}%
     {.0em}%\usepackage{relsize}
     {\filright\boxedsection}%
 \titleformat{\subsection}[hang]%
     {\usekomafont{subsection}}%
     {}%
     {.0em}%
     {\filright\boxedsubsection}%
 \titleformat{\subsubsection}[hang]%
     {\usekomafont{subsubsection}}%
     {}%
     {.0em}%
     {\filright\boxedsubsubsection}%

\usepackage{fancyhdr}                  

%========================== Fancy Header Style Options =============================%
\pagestyle{fancy}                       % Sets fancy header and footer
\fancyfoot{}                            % Delete current footer settings
%\renewcommand{\chaptermark}[1]{         % Lower Case Chapter marker style
%  \markboth{\chaptername\ \thechapter.\ #1}}{}} %
%\renewcommand{\sectionmark}[1]{         % Lower case Section marker style
%  \markright{\thesection.\ #1}}         %
\fancyhead[LE,RO]{\bfseries\thepage}    % Page number (boldface) in left on even
% pages and right on odd pages
\renewcommand{\chaptermark}[1]{\markboth{\MakeUppercase{\thechapter.\ #1}}{}}
\fancyhead[RE]{\bfseries\nouppercase{\leftmark}}      % Chapter in the right on even pages
\fancyhead[LO]{\bfseries\nouppercase{\rightmark}}     % Section in the left on odd pages

\let\headruleORIG\headrule
\renewcommand{\headrule}{\color{black} \headruleORIG}
\renewcommand{\headrulewidth}{1.0pt}
\usepackage{colortbl}
\arrayrulecolor{lightgray}

\fancypagestyle{plain}{
  \fancyhead{}
  \fancyfoot[C]{\thepage}
  \renewcommand{\headrulewidth}{0pt}
}

\usepackage{algorithm}
\usepackage[noend]{algorithmic}

%================================== Clear Header ==================================%
\makeatletter
\def\cleardoublepage{\clearpage\if@twoside \ifodd\c@page\else%
  \hbox{}%
  \thispagestyle{empty}%              % Empty header styles
  \newpage%
  \if@twocolumn\hbox{}\newpage\fi\fi\fi}

\makeatother


%================================ RENEWs COMMANDS =================================%
\usepackage{multirow}
\usepackage{slashbox}

\renewcommand{\epsilon}{\varepsilon}
% centered page environment
\newenvironment{vcentrepage}
{\newpage\vspace*{\fill}\thispagestyle{empty}\renewcommand{\headrulewidth}{0pt}}
{\vspace*{\fill}}
\renewcommand{\figurename}{Figure}
%\usepackage[small,bf]{caption}
%\addtokomafont{caption}{\small}
%\setkomafont{captionlabel}{\small\bfseries}
\setcapindent{0em}
\usepackage{ifpdf}
\renewcommand{\thefootnote}{\roman{footnote}}
\def\sectionautorefname{Section}
\def\subsectionautorefname{Section}
\def\subsubsectionautorefname{Section}
\usepackage{pdflscape}
\newenvironment{mysidewaystable}[1][htp]{\begin{sidewaystable}[#1]}{\end{sidewaystable}}
\renewcommand{\topfraction}{0.7}	% max fraction of floats at top
\renewcommand{\bottomfraction}{0.75}	% max fraction of floats at bottom

%=================== Parametres for TEXT pages (not float pages)====================%
\setcounter{topnumber}{2}
\setcounter{bottomnumber}{1}
\setcounter{totalnumber}{3}     % 2 may work better
\setcounter{dbltopnumber}{2}    % for 2-column pages
\renewcommand{\textfraction}{0.05}	% allow minimal text w. figs
%   Parametres for FLOAT pages (not text pages):
\renewcommand{\floatpagefraction}{0.6}	% require fuller float pages
% N.B.: floatpagefraction MUST be less than topfraction !!
\usepackage[title,toc]{appendix}


%================================ add page to refs ==================================
\renewcommand*{\backref}[1]{}
\renewcommand*{\backrefalt}[4]{%
    \ifcase #1 (No citado.)%
    \or        (Citado en la página~#2.)%
    \else      (Citado en la páginas~#2.)%
    \fi}

\makeatletter
\newcommand*\bigcdot{\mathpalette\bigcdot@{.5}}
\newcommand*\bigcdot@[2]{\mathbin{\vcenter{\hbox{\scalebox{#2}{$\m@th#1\bullet$}}}}}
\makeatother

\usepackage{typearea}
\usepackage{color}
\usepackage{mwe}
\usepackage{listings}    
\usepackage{etoolbox}

\pgfdeclaredecoration{jiggly}{step} %from https://tex.stackexchange.com/questions/438530/drawing-a-rough-surface-with-tikz
  {
    \state{step}[width=+\pgfdecorationsegmentlength]
    {\pgfmathsetmacro{\delta}{rand*\pgfdecorationsegmentamplitude}
      \pgfpathlineto{
        \pgfpointadd
        {\pgfpoint{\pgfdecorationsegmentlength}{0pt}}
        {\pgfpointpolar{90-\pgfdecoratedangle}
            {\delta}}
      }
    }
    \state{final}
    {
      \pgfpathlineto{\pgfpointdecoratedpathlast}
    }
  }

  %from https://tex.stackexchange.com/questions/123158/tikz-using-the-ellipse-command-with-a-start-and-end-angle-instead-of-an-arc
  \tikzset{
    partial ellipse/.style args={#1:#2:#3}{
        insert path={+ (#1:#3) arc (#1:#2:#3)}
    }
}

%from https://tex.stackexchange.com/questions/107171/manipulating-isosceles-triangle
\tikzset{
    rubberduck/.style={
        draw=gray,
        shape=isosceles triangle,
        fill=gray,
        minimum height=0.7cm,
        minimum width=0.5cm,
        shape border rotate=#1,
        isosceles triangle stretches,
        inner sep=0pt,
    },
    rubber/.style={rubberduck=+90},
    ducky/.style={rubberduck=-90}}