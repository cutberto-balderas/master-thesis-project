\chapter{Muestra y técnicas experimentales.}
\label{chap:techniques-and-sample}
\textit{En este capitulo discutiremos los principales aspectos de la muestra estudiada y las técnicas utilizadas para la caracterización de la misma.}
\vfill
\minitoc
\newpage

\section{Descripción de la muestra.}
\label{sec:chap3-sample-description}
\newpage

\section{Espectroscopia de Reflectancia Diferencial}
\label{sec:chap3-rds}
La técnica de Espectroscopia de Reflectancia Diferencial \textbf{(RDS)} es
utilizada en el estudio de dispositivos y materiales, haciendo uso de la
\textit{anisotropía}, la cual es la propiedad que describe como un material
puede tener diferentes respuestas dependiendo de la 
\\dirección en la que es examinada. En la \textbf{RDS}, se estudia la anisotropía
óptica del sistema, 

\newpage

\section{Espectroscopia Raman}
\label{sec:chap3-raman}
\newpage

\section{Microscopia de Fuerza Atómica}
\label{sec:chap3-afm}
\newpage

\section{Microscopía Óptica de Barrido de Campo Cercano}
\label{sec:chap3-nsom}
\newpage