\chapter{Muestra y técnicas experimentales.}

\label{chap:techniques-and-sample}
\textit{En este capitulo discutiremos los principales aspectos de la muestra estudiada y las técnicas utilizadas para la caracterización de la misma.}
\vfill
\minitoc
\newpage

\section{Descripción de la muestra.}
\label{sec:chap3-sample-description}
\newpage

\section{Espectroscopia de Reflectancia Diferencial}
\label{sec:chap3-rds}
La técnica de Espectroscopia de Reflectancia Diferencial \textbf{(RDS)} es utilizada en el estudio 
de dispositivos y materiales, haciendo uso de la \textit{anisotropía}, la cual es la propiedad que 
describe como un material puede tener diferentes respuestas dependiendo de la dirección en la que es examinada, 
en este caso se debe a la \textit{anisotropía óptica}, donde observamos los cambios en la respuesta óptica 
del sistema estudiado, utilizando las propiedades de simetría y la polarización de la luz, para obtener 
una diferencia en la reflectancia del sistema observándolo de diferentes direcciones.

En el caso de los materiales semiconductores, nos aprovechamos de las propiedades
de simetría de los cristales observando dos direcciones cristalográficas, se elimina
la contribución del bulto o el cuerpo principal del semiconductor provocando que sea \textit{isotrópica}, 
obteniendo que la respuesta dependa solamente la superficie, haciendo que la \textbf{RDS} sea una 
técnica utilizada para el estudio de superficies.\cite{Aspnes1985} La respuesta o espectro de la 
\textbf{RDS} tiene la siguiente forma:

\begin{equation}
    \label{eqn:ch3-rds-eqn}
    \dfrac{\Delta R}{R} = 2\dfrac{R_{\alpha}-R_{\beta}}{R_{\alpha}+R_{\beta}}
\end{equation}

Siendo $ \alpha $ y $ \beta $ las direcciones cristalográfica para las cuales se obtuvo la
reflectancia del material, siendo la diferencia de estos la que nos da la información de solamente la 
superficie.

*Colocar figura del sistema*

El sistema utilizado para las mediciones tiene como fuente de iluminación una lampara de Xenón la cual es
enfocada hacia el monocromador con el uso de un arreglo de espejos, del cual sale la luz difractada dirigida a
otro arreglo de espejos los cuales dirigen la luz a la parte del sistema que controla su polarización, 
un prisma polarizador y un modulador fotoelástico que al pasar por ellos, obtenemos un haz de luz con 
una cierta polarización y que se estará modulando entre dos estados perpendiculares entre si, incidiendo sobre 
la muestra para reflejarse hacia un tubo fotomultiplicador.\cite{LastrasMartnez1993}

\newpage

\section{Espectroscopia Raman}
\label{sec:chap3-raman}
La Espectroscopia Raman es una técnica que se aprovecha del \textit{efecto Raman}, el cual describe la interaccion
entre los fotones, el material que estamos estudiando y la forma en el que este vibra cuando los fotones incidan 
sobre una molécula, pudiéndose presentándose algún modo vibracional, siendo característico de la muestra, 
ya que esta estrechamente relacionado con la estructura molecular del material, resultando en la dispersión de 
fotones con diferentes frecuencias.

Cuando un fotón choca contra una molécula puede ocurrir alguno de los siguientes casos:
    \begin{itemize}
        \item Que el resultado sea un choque elástico, queriendo decir que no perderá energía por medio de
        efectos vibracionales, dando como resultado un fotón dispersado con la misma frecuencia. Esto se conoce 
        como efecto Rayleigh.

        \item Que el choque resultante sea inelástico, lo que provoca que el fotón dispersado tenga una frecuencia 
        diferente a la del incidente, teniendo el caso de una mayor frecuencia, provocando una ganancia en la energía 
        con respecto al incidente y donde el fotón dispersado tiene menor frecuencia que el incidente, indicando que 
        ocurrió una perdida de la energía por procesos vibracionales. Estos son nombrados efecto Anti-Stokes y efecto 
        Stokes respectivamente.
    \end{itemize}

*Colocar figura de diagrama de energía y efecto Rayleigh*

Por consecuencia, el efecto Stokes es utilizado para observar el \textit{efecto Raman}, siendo mas probable que
este suceda porque sus fotones dispersados tienen una energía menor que su contraparte del efecto Anti-Stokes.

En el caso de los semiconductores, debido a la existencia de la red cristalina, el efecto Raman nos puede dar información 
sobre la cristalinidad de la muestra, debido a que cada modo vibracional tiene una frecuencia para los fotones dispersados 
entre estos mas se alejen o apeguen de este valor nos da una noción de que tan cristalina es la muestra. Otra propiedad
importante es el entender el estrés al que esta sometido la muestra, por el desplazamiento de la respuesta del material 
en contraste con que no presente estrés.

*Posiblemente colocar diagrama del sistema y descripción*
\newpage

\section{Microscopia de Fuerza Atómica}
\label{sec:chap3-afm}
La Microscopia de Fuerza Atómica (AFM), es una técnica que es capaz de medir la superficie de un material a nivel 
nanométrico utilizando como principio las fuerzas de interaccion atractivas y repulsivas entre la punta del instrumento 
con el material. Para que la medición sea correcta, la punta debe tener una terminación en una cantidad de átomos 
pequeña, que al momento de experimentar una fuerza, esta provocara una deflexión en el cantiléver sobre el cual esta 
montada.\cite{Binnig1986}

*Insertar imagen punta-muestra Binning*
*Explicar los diferentes tipos de técnica o usar solo el usado*
*¿Vale la pena el juntar AFM y NSOM por el equipo usado?*
\section{Microscopía Óptica de Barrido de Campo Cercano}
\label{sec:chap3-nsom}
\newpage